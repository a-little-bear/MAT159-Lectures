\documentclass[11pt, cyan, night, 1in]{LatexTemplate/hw}

\def\course{MAT159}
\useclspackage{csc240}

\begin{document}

\section{Review}

Riemann Integral. $f\in R[a,b]$
\begin{itemize}
    \item $[a,b]$ is a closed and bounded
    \item $f$ is bounded on $[a,b]$
\end{itemize}

\section{Improper Integral}

\begin{itemize}
    \item Unbounded Integral
    \item Unbounded funciton
\end{itemize}

\section{Unbounded inteval}
\begin{itemize}
    \item $f: [a,+\infty)$
    \item $\forall b\in(a,+\infty), f\in R[a,b]$
\end{itemize}

If $\lim_{b\to\infty}\int_a^b f(x)\D x$ exists, then we can define \[\int_a^{+\infty}f(x)\D x:= \lim_{b\to+\infty}\int_a^bf(x)\D x.\]

Similarly we can define \[\int_{-\infty}^b f(x)\D x.\]

\neweg{
    $\int_0^{+\infty}\frac{1}{1+x^2}\D x$

    Then, $\forall b>0. \frac{1}{1+x^2}\in R[a,b]$

    Moreover, \begin{align*}
        \lim_{b\to+\infty}\int_0^b\frac{1}{1+x^2}\D x &= \lim_{b\to+\infty}[\arctan x]_0^b\\
        &=\lim_{b\to+\infty} \arctan b\\
        &=\frac{\pi}{2}
    \end{align*}

    Hence $\lim_0^{+\infty} \frac{1}{1+x^2}\D x = \frac{\pi}{2}$.
}

\newr{\hfill

    \begin{itemize}
        \item By \underline{FTC}. Rewriting $\lim_{b\to+\infty}\int_a^b f(x)\D x$ exists is equivalent of rewriting $\lim_{b\to+\infty} F(b)$ exists, where $F(b)=\int_a^b f(x)\D x$.
        \item By \underline{Sequence / Series}, $\lim_{b\to+\infty}\int_a^b f(x)\D x$ exists is equivalent of rewriting that for all $a_n, (a_0=a)$ monotonically increasing to positive infinity, \[\sum_{n=0}^\infty\int_{a_n}^{a_{n+1}}f(x)\D x < \infty.\]
        
        However, if $f(x)\ge 0$, then $\exists$ is sufficient.
    \end{itemize}
}

\neweg{
    $\int_1^{+\infty} \frac{1}{x^\la}\D x\quad \la>0$

    By FTC, $\forall b>1$, \[\int_1^b \frac{1}{x^\la}\D x = \begin{cases}
        \frac{1}{1-\la}\cd x^{1-\la}\bigg|_1^b & \la\ne 1\\
        \ln x\bigg|_1^b & \la=1
    \end{cases}\]

    Hence, \[\int_a^{+\infty}\frac{\D x}{x^\la}=1 \quad (\la >1),\] and is undefined when $\la\le 1$
}

\newr{
    This is a continuous analogue of p-harmonic series
}

\neweg{
    $\int_0^{+\infty} \cos x \D x$

    By FTC, $\forall b>0$, \[\int_a^b \cos x\D x = \sin b \quad \T{ periodic},\]

    Since \(\lim_{b\to+\infty}\sin b\) does \underline{\tbf{NOT}} exists, hence \(\lim_{0}^{+\infty} \cos x \D x\) is indefined.
}

\subsection{Criteria}

\begin{itemize}
    \item Cauchy Sequence Properties
    \indenv{
        $\int_a^{+\infty} f(x)\D x$

        $\forall \ep>0. \exists A>0$ s.t. \begin{align*}
            \forall b_1,b_2>A, &\abs{\int_a^{b_1}f(x)\D x - \int_a^{b_1} f(x)\D x} < \ep\\
            &\equiv \abs{F(b_1)-F(b_2)}<\ep.
        \end{align*}
    }
    \item Abel and Dirichlet 
    \indenv{
        $\int_a^{+\infty}f(x)g(x)\D x$

        \underline{Combo Abel}
        \indenv{
            \begin{enumerate}
                \item $\int_a^b f(x)\D x < \infty$
                \item $g$ is monotonic and bounded
            \end{enumerate}
        }
        \underline{Combo Dirichlet}
        \indenv{
            \begin{enumerate}
                \item $\abs{\int_a^b f(x)\D x} < M$
                \item $g$ monotonically decreasing to 0.
            \end{enumerate}
        }

        \neweg{
            $\int_1^{+\infty}\frac{\sin x}{x}\D x$

            $\int_1^{+\infty}f(x) g(x)\D x$, where $f(x)=\sin x$ and $g(x)=\frac{1}{x}$.

            \begin{enumerate}
                \item $\forall b10$, $\abs{\int_1^b\sin x\D x} = \abs{1 - \cos b}\le 2$
                \item $\frac{1}{x}$ monotonically decreasing to 0 when $x\to+\infty$.
            \end{enumerate}

            So Dirichlet works.
        }
    }
\end{itemize}

\newr{(Warning)\hfill

    If $f\in R[a,b]$, then $|f|\in R[a,b]$. Moreover $\abs{\int_a^b f(x)\D x}\le \int_a^b \abs{f(x)}\D x$.

    On the other hand, it is not true that \[\int_1^{+\infty} |f(x)|\D x <\infty \iimplies \int_1^{+\infty} f(x)\D x < +\infty,\]

    (Counter)-example:
    \[\int_1^{+\infty}\frac{|\sin x|}{x}\D x\quad\T{ is undefined}\]

    \underline{Actually,} \[\abs{\sin x}\ge \sin^2 x = \frac{1-\cos 2x}{2}\]

    $\int_1^{+\infty}\frac{1-\cos 2x}{2} / x \D x = \int_1^{+\infty}\frac{1-\cos 2x}{2x}\D x$

    \begin{itemize}
        \item $\int_1^{+\infty}\frac{\cos 2x}{2x}\D x <\infty$ (By Dirichlet)
        \item $\int_1^{+\infty}\frac{1}{2x}\D x = \infty$
    \end{itemize}

    These two imply $\int_1^{+\infty}\frac{1-\cos 2x}{2x}\D x$ is undefined.
}

\section{Unbounded Functions}

$f:[a,b)\to\R$ is unbounded in $[b-\et,b)$

$f\in R[a,b-\et], \forall \et\in(0,b-a)$

\fig{img/2024-04-08-09-58-58.png}

If $\lim_{\et\to0^+}\int_a^{b-\et}f(x)\D x$ exists, then \[\int_a^b f(x)\D x :=\lim_{\et\to 0^+}\int_a^{b-\et}f(x)\D x\]

Similarly \[\int_a^b f(x)\D x \quad\T{ if $f(x)$ is unbounded on $(0, a+\et)$}\]

\neweg{\hfill

    $\int_0^1\frac{1}{x^{\la}}\D x\quad \la>0$

    $\int_{\et}^1\frac{1}{x^\la}\D x = \begin{cases}
        \frac{x^{1-\la}}{1-\la} \bigg|_{\et}^1 & \la\ne 1\\
        \ln x\bigg|_{\et}^1 & \la=1
    \end{cases}$

    Hence, \[\int_0^1\frac{1}{x^\la}\D x = \begin{cases}
        \frac{1}{1-\la} & \T{ if } 0<\la<1\\
        \T{undefined otherwise} &
    \end{cases}\]

}

\newr{
    You can often use Tyalor Expansion near the singularity.
}


\neweg{
    $\int_0^1\frac{\D x}{\sqrt[3]{x\bra{e^x-e^{-x}}}}$

    Let \begin{align*}
        g(x)&=x\bra{e^x-e^{-x}}\\
        &= x\bra{1+x+\frac{x^2}{2}-\bra{1-x+\frac{x^2}{2}}}\\
        &= x(2x + o(x^2)) \approx 2x^2
    \end{align*}    
}



\neweg{
    Consider the pendulum

    \fig{img/2024-04-08-10-18-49.png}

    \[\int_0^{x_0}\frac{\D x}{\sqrt{\cos x - \cos x_0}}\]

    $\cos x - \cos x_0 \approx -(x-x_0) \sin x_0$

    $\sin x_0 \neq 0$ \quad when $0<x_0<\pi$

    Hence \[\int_0^{x_0} \frac{\D x}{\sqrt{\cos x - \cos x_0}}\le C\int_0^{x_0}\frac{\D x}{\sqrt{x_0 - x}} < \infty\]


}

\newr{
    Improper integral for both unbounded interval and unbounded function:

    \fig{img/2024-04-08-10-29-48.png}

    $\int{0}^{+\infty}f(x)\D x$

    If we choose an $A$ that is large enough, and pass $A$ to positive infinity.

    However,

    \fig{img/2024-04-08-10-30-02.png}

    \[\int_{-\infty}^{+\infty} f(x)\D x\]

    \[\lim_{\substack{b\to+\infty\\ \tilde{b}\to-\infty}}\int_{\tilde{b}}^b f(x)\D x\]

    \tbf{\underline{Warning}}. $b$ and $\tilde{b}$ are independent of each other.

    \fig{img/2024-04-08-10-34-26.png}

    \[\int_b^c f(x)\D x := \lim_{\et\to0}\int_b^{a-\et}f(x)\D x + \lim_{\tilde{\et}\to0}\int_{a+\tilde{\et}}^{c}f(x)\D x\]

    Here $\et$ and $\tilde{\et}$ are independent of each other.
}

\neweg{

    \fig{img/2024-04-08-10-36-38.png}

    \begin{align*}
        \int_{-1}^1\frac{1}{x}\D x &:= \lim_{\et\to 0}\int_{-1}^{\et}\frac{1}{x}\D x + \lim_{\tilde{\et}\to 0}\int_{\tilde{\et}}^1\frac{1}{x}\D x\\
        &=\lim_{\et\to 0}\ln(-x)\bigg|_{-1}^{-\et}+\lim_{\tilde{\et}\to 0}\ln x\bigg|_{\tilde{\et}}^1\\
        &=\lim_{\et\to 0}\lim_{\tilde{\et}\to 0}\ln\bra{\frac{\et}{\tilde{\et}}}\\
        &=\infty
    \end{align*}

    But, if we force that $\et=\tilde{\et}$, then the limit exists, and is equal to 0. The above value is called the \underline{Cauchy Principle Value} of improper integral.

    We should write this as \[\T{v.p.}\quad\int_{-1}^1\frac{1}{x}\D x = 0\]

}

\neweg{
    \[\int_0^{+\infty}e^{-x^2}\D x = \frac{\sqrt{\pi}}{2}\]

    First, it is an improper integral, and we can see the speed it goes to infinity is much slower than any polynomial reverse:

    \[\lim_{x\to+\infty}\frac{e^{-x^2}}{\frac{1}{x^2}}=0\]

    since $\int_1^{+\infty}\frac{1}{x^2}<\infty\iimplies \int_1^{+\infty}e^{-x^2}\D x < +\infty \iimplies \int_0^{+\infty}e^{-x^2}\D x < \infty$.

    Second, FTC does not work, it is not integralble in finite terms, gg.

    Lets do some preparations:
    \[J_n:=\int_0^{\frac{\pi}{2}}\sin^n x\D x\]

    Then \begin{align*}
        J_n &= \int_0^{\frac{\pi}{2}}\sin^n x\D x\\
        &= \int_0^{\frac{\pi}{2}}\sin^{n-1}x\D(-\cos x)\\
        &= \sin^{n-1}x\cos x\bigg|_0^{\frac{\pi}{2}} + (n-1)\int_0^{\frac{\pi}{2}}\sin^{n-2}x\cos^2 x\D x\\
        &=(n-1)\int_0^{\frac{\pi}{2}}\cos x \sin^{n-2}x\cd\cos x \D x\\
        &=(n-1)\cd\bra{\int_0^{\frac{\pi}{2}}\sin^{n-2}x\D x - J_{n-2}}\\
        &=(n-1)J_{n-2} - (n-1)J_n\\
        J_n &= \frac{n-1}{n}J_{n-2}\\
        J_0 &= \frac{\pi}{2}\\
        J_1 &= 1
        \alt{By induction,}
        J_n &= \begin{cases}
            \frac{(n-1)!!}{n!!}\cd\frac{\pi}{2} & n\T{ is even}\\
            \frac{(n-1)!!}{n!!} & n\T{ is odd}
        \end{cases}
    \end{align*}

}

\section{Wallis}

\newcl{1}{
    \[\lim_{n\to\infty}\int_0^{\frac{\pi}{2}}\sin^n x\D x=0\]
}

\newp{[Fake proof]\hfill

\[\lim_{n\to\infty}\int_0^{\frac{\pi}{2}}\sin^n x\D x = \int_0^{\frac{\pi}{2}}\lim_{n\to\infty} \sin^n x\D x = \int_0^{\frac{\pi}{2}}0\D x = 0\]

}

\newp{[Real proof]\hfill

Consider \[J_{2n+1} < J_{2n} < J_{2n-1}\]

i.e. \[\frac{(2n)!!}{(2n+1)!!}<\frac{(2n-1)!!}{(2n)!!}\cd\frac{\pi}{2}<\frac{(2n-2)!!}{(2n-1)!!}\]

\[\lim_{n\to\infty}\bra{\frac{\pi}{2}\cd\frac{(2n-1)!!}{(2n)!!}}^2=\frac{(2n)!!}{(2n+1)!!}\cd\frac{(2n-2)!!}{(2n-1)!!}\]

Which implies by Wallis formula, $\frac{\pi}{2}=\lim$...

}

To be finished : )




\end{document}
