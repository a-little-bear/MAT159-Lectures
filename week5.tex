% \documentclass[11pt, white, night, 1in]{alittlebear}

% \def\course{MAT159: Analysis II}
% \def\headername{Week 5 Lecture}
% \def\name{Joseph Siu}

% \extractfootnote{proposition}

% \begin{document}

% \coverpage[clsfiles/stars]


% \newtbox[Review]{
%     \begin{itemize}
%         \item Definition of $f\in \cal{R}[a,b]$
%         \item Criteria for $f\in\cal{R}[a,b]$:

%         1) 2) 3) 4) 5)
%         \item Proposition of $f\in\cal{R}[a,b]$
%         \begin{itemize}
%             \item Linearity: $\forall f,g\in\cal{R}[a,b]$. $\forall\al,\be\in\R$. $\al f +\be g\in\cal{R}[a,b]$ and $\int_{a}^{b}(\al f +\be g) = \al\int_{a}^{b}f + \be\int_{a}^{b}g$
%         \end{itemize}
%     \end{itemize}
% }

% \newr{
%     Riemann integral: $\cal{R}[a,b]\stackrel{\int_a^b}{\to}\R$, $f\to\int_{a}^{b}f(x)\D x$ can be seen as a linear operator. 
% }

% \newr{
%     If $g\in C[a,b], f\in\cal{R}[a,b]$, then $\cal{R}[a,b]\stackrel{\int_a^b\cd g\D x}{\to}\R$, $f\to\int_{a}^{b}f(x)g(x)\D x$ is a linear operator.

%     \neweg{
%         $L:\cal{R}[a,b]\to\R$ is well-defined and linear.
%     }

%     \begin{itemize}
%         \item \underline{Composition} $g\circ f$.
%         $$\begin{array}{c|c|c}&f\in\mathcal{C}[a,b]&f\in\mathfrak{R}[a,b]\\\hline g\in\mathcal{C}[a,b]&\T{Yes}&\T{Yes}\\\hline g\in\mathfrak{R}[a,b]&\T{No}&\T{No}\\\hline\end{array}$$
%         \item \underline{Monotonicity}
%     \end{itemize}
% }

% \newpp{1}{
% For any $f,g\in\cal{R}[a,b]$, $f\leq g$ on $[a,b]$ implies $\int_{a}^{b}f\leq\int_{a}^{b}g$.

% \newp{
%     Take a sequence of\footnote{converge to the integral} marked partitions $(\Ga_n,\et)\in\Om^*[a,b]\st\norm{\Ga_n}\to0$. Now, \begin{align*}
%         \sum_{k=0}^m f(\et_{n_k})\De x_k&\leq\sum_{k=0}^m g(\et_{n_k})\De x_k\\
%         \xdownarrow{2} n\to\infty\quad&\quad\left\downarrow n\to\infty\right.\\
%         \xrightarrow[n\to\infty]{}\int_{a}^{b}f(x)\D x&\leq\int_{a}^{b}g(x)\D x\xleftarrow[n\to\infty]{}
%     \end{align*}
% }
% }

% \newco{1}{
%     $f\in\cal{R}[a,b],f\geq0$, then $\int_{a}^{b}f\geq0$. 

%     (we compare the average as integral)
% }

% \np

% \newr{
% If $f\in\cal{R}[a,b], g\leq f$, $g$ is bounded, we cannot ensure $g$ is integrable.
% \neweg{
%     $f=2,g=\begin{cases}
%         1&\T{if }x\in\Q\\
%         0&\T{if }x\notin\Q
%     \end{cases}$ on $[0,1]$.
% }
% }

% \newtbox[Additivity]{
%     Observation: If $f:\cal{R}[a,b]$, $\forall \overline{a}\leq\overline{b}\st[\overline{a},\overline{b}]\subset[a,b]$, then $f\in\cal{R}[a,b]$.
% }

% \newpp{2}{$f\in\cal{R}[a,b]$, $\forall c_1,c_2,c_3\in[a,b]$, $\int_{c_1}^{c_3}f=\int_{c_1}^{c_2}f+\int_{c_2}^{c_3}f$.

% \newh{
% We did not assume $c_1\leq c_2\leq c_3$ due to the convention $\int_b^a=-\int_a^b$.
% }

% }

% lazy to type the later part... 




% \end{document}